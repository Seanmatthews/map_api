\documentclass{article}
\usepackage[utf8]{inputenc}
\usepackage{todonotes}
\usepackage{a4wide}

\title{dmap design document}
\author{Titus Cieslewski}
\date{Spring 2014}

\setlength{\parindent}{0mm}
%\setlength{\parskip}{3mm}

\begin{document}

\maketitle

\section{Overview}

The dmap should provide a mean for robots to collaborate on maps on a large
scale. It is designed with collaboration on a pose graph in mind, but pose
graph and the backend are separated, allowing using the dmap for other
map-related tasks.

The envisioned long-term goal is to develop a Peer-to-peer database system
(PDBS) geared towards mapping purposes.

\section{Application-to-database mapping, table types}

Applications acces the dmap by extending a table interface class. What
fields the application would want to use is to be defined in the virutal
method {\tt define()}, using calls to {\tt addField()}. The mechanisms used to
ensure synchronized definition of tables are specified in section 
\ref{define-trans}.

Two types of table interfaces can be chosen: Table interfaces that allow
updating items (CRU tables) and table interfaces that don't (CR tables). 

Both expose the the method {\tt std::shared\_ptr$<$Revision$>$ getTemplate()},
which returns a revision object which allows the application to set the fields
of an item to be inserted into the database.

Any operation on the table is done via transactions. Insertion and update takes
a populated revision as argument, reading returns one. Table lookups can for
now only be done via 128 bit Hashes, which are the primary keys of tables.

\subsection{Structure of a {\tt map\_api::Revision}}

A revision object is an extended protocol buffer message which contains the
name of the table that the object belongs to, the type (CR or CRU),
and repeated key-type-value
triples. These contain the field information of the row represented by the
revision.

\subsection{Default fields}

All tables have per default two fields: ID and owner, both of which are
currently 128 bit hashes stored in string form. ID's are the primary key and
thus need to be unique. Per default, they are randomly generated.

\subsection{Revision history for updatable tables}

Tables with ability to update need to carry a revision history in order to be
able to perform meaningful transactions.
Instead of containing the latest revision themselves, CRU table rows contain
a reference to their latest revision each. More precisely, the reference is to
an object with the following information:

\begin{itemize} \itemsep0em
  \item The ID of the corresponding CRU table row
  \item The revision
  \item A precise timpestamp of the revision
  \item A reference to the same object type, representing the previous revision
\end{itemize}

In the implementation, these objects are stored in a revision history table
which is a CR table and has the same name as the table it versions, suffixed
with {\tt \_history}.

\section{Transactions}\label{transactions}

Queries to the dmap are handled by transactions, as previously specified in
the Map Manager API by Stéphane.

\subsection{Transaction walkthrough}

\begin{enumerate} \itemsep0em
  \item When a transaction is initialized, the timestamp is taken. This allows
    for the transaction to hold a view on the database as it is at this
    particular instant.
  \item Write queries to the API are queued, not journaled and executed as in 
    SQL. This reduces overhead in case of conflict. Read queries are executed
    directly, but the returned revisions are always the latest revisions before
    the transaction view timestamp.
  \item When the transaction is filed for commit, a global reader lock for the
    database is acquired \todo{tcies: per-item locks, deadlock prevention}
  \item All modified objects are checked for conflict. In case of conflict,
    the transaction is aborted, but the application may retrieve the queue of
    queries in order to handle the conflict and attempt to re-apply.
  \item If the conflict handler succeeds or there are no conflicts, the lock
    is upgraded to writer lock and the
    queries are executed.
\end{enumerate}

\subsection{Query types}

\begin{itemize} \itemsep0em
  \item {\bf Insert queries} take a populated revision as argument and return
    a hash id. They only lead to conflict if the assigned hash ID is
    already present in the local database. It is in general assumed that insert
    conflicts don't happen.
  \item {\bf Read queries} take a hash id as argument and
    always return the latest revision from before the
    beginning of the transaction. \todo{Way to notify about outdated data?}
  \item {\bf Update queries} take a hash id and a revision as argument. 
    When update queries are queued, the latest
    revision before the update is referenced. If there exists a later revision 
    at commit time, there is a conflict.
\end{itemize}

\subsection{Transaction involved in defining tables}\label{define-trans}

Before applications use tables, they need to define them. This involves
synchronizing the table schema with other nodes that use the table, or creating
the table if it doesn't exist yet. The latter could be subject to a race
condition, so needs to be synchronized. This happens in a transaction which is
completely hidden away from the user, that is, the choice is not given to bind 
the
definition of a table with other queries into a transaction for the time being
\todo{tcies: possible future feature: couple definition of multiple tables, or 
just make it generally available for inclusion in transactions}. As we don't
assume that tables will change over time, the transaction is a bit simpler than
a generic transaction described above:

\begin{enumerate}
  \item Lock a global reader lock specific to the definition of tables
  \item Check if a table with the defined name already exists. If yes, verify
    the schemata are congruent or fail.
  \item If the schemata are different, upgrade to writer lock and create new
    table
\end{enumerate}

\section{First implementation simplifications}

\begin{itemize}
  \item Not using reader-writer locks for now but just simple locks
\end{itemize}

\end{document}
